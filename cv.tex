\documentclass{mycv}

\name{Daniel Ko \small{(US Citizen)}}
\email{danielhbko@gmail.com / ko28@wisc.edu}
\homepage{https://ko28.github.io}
\github{ko28}
\linkedin{daniel-ko1}

\begin{document}

\maketitle%

\section{Professional \\ Experience}
\subsection{TDS Telecom}[Madison, WI]
\begin{positions}
  \entry{Software Engineering Intern}{Sept 2019~--~Present}
\end{positions}

\begin{itemize}
  \item {
  Addressed regular production bugs and improvements in enterprise \textbf{Java}
applications using \textbf{JIRA} to prioritize tasks and utilized \textbf{CI/CD} pipelines for
rapid development and deployment.
  }
  \item {
  Developed \textbf{Python} script that used the \textbf{ArcGIS API} to automate uploading
service definition files to and from a remote server, increased upload speeds
by 80\%.
}
  \item {
  Developed, refactored, and upgraded legacy \textbf{Perl} web applications and scripts
on an Operations Support Systems (NetExpert) server, which interacted with
internal \textbf{RESTful} services and \textbf{SQL} servers. Web applications used by more than 1,000
field technicians.
  }
\end{itemize}

\subsection{Juni Learning}[Remote]
\begin{positions}
  \entry{Computer Science Instructor}{Sept 2019~--~Jan 2019}
\end{positions}

\begin{itemize}
  \item {
  Individually taught 8 students aged 8-16 \textbf{Scratch}, \textbf{Python}, and \textbf{Java} (AP Computer Science). 
  }
  \item{
  Successfully taught trial sessions resulting in more than 50\% enrollment rate.
  }
\end{itemize}

\subsection{NASA 2019 Student Launch Initiative}[Madison, WI]
\begin{positions}
  \entry{Chief Hardware Engineer}{Sept 2018~--~June 2019}
\end{positions}

\begin{itemize}
  \item Developed full stack application for our project: \textit{Measurement of Aerosols in Lower Atmosphere Using Optical Detection}.  
  \item {
  Included communication between payload and telemetry module to ground computer using a \textbf{XBee} module using serial. Ground computer had a \textbf{Python} backend which used NumPy and pandas to parse through and analyze telemetry and payload data. \textbf{Bootstrap} frontend for interactive dashboard with live 3D model tracking the gyroscope sensor.
  }
\end{itemize}

\subsection{Yu Lab (University of Wisconsin - Madison)}[Madison, WI]
\begin{positions}
  \entry{Research Intern}{Feb 2018~--~Sept 2018}
\end{positions}

\begin{itemize}
  \item{
  Researched various genes in the \textit{Aspergillus} genus and their effects on sporic life cycle. 
  }
\end{itemize}

\subsection{West High School}[Madison, WI]
\begin{positions}
  \entry{AP Calculus \& Algebra Teaching Assistant}{Sept 2017~--~June 2018}
\end{positions}

\begin{itemize}
  \item{
  Spent 1 year extensively helping students with classwork, homework, and test reviews. 
  }
\end{itemize}

\section{Education}

\subsection{University of Wisconsin - Madison}[Madison, WI]
\vspace{-\parskip}%
\begin{itemize}[label={}]
  \item B.S.\ in Computer Science \printdate{Jan 2020~--~Dec 2022}
  \item 3.95/4.00 Cumulative GPA
  \item Relevant Coursework: Discrete Mathematics, Linear Algebra, Multivariable Calculus, Data Structures, Algorithms  
\end{itemize}

\section{Skills}

\begin{description}
  \item[Programming] Java, Python, Perl, C/C++, SQL, Bash, HTML, CSS, Javascript
  \item[Tools] Vim, Tmux, Git, JIRA, Jenkins, *nix, \LaTeX 
  \item[Languages] English, Korean
  \item[Miscellaneous] Red Cross Adult and Pediatric First Aid/CPR/AED Certification
\end{description}

\section{Awards}
\begin{itemize}
\item{ 
Foreign Language and Area Studies (FLAS) Fellowship, U.S. Department of Education \printdate{2020}
}
    \item {STEM Engagement Award (2nd), NASA Student Launch Initiative \printdate{2019}}
    \item {Altitude Award (3rd), NASA Student Launch Initiative \printdate{2019}}
    \item {Social Media Award (2rd) , NASA Student Launch Initiative \printdate{2019}}
    \item {1st Place, Rocket for Schools Competition (cooperation with Gilroy and Ane
Lab) \printdate{2018, 2019}}
  \item National Finalist, Team America Rocketry Challenge \printdate{2018}
 
\end{itemize}
\thispagestyle{empty} 
\end{document}
